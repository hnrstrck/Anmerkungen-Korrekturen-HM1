% Hinweise zur Korrektur der HM1-Abgaben
% **************************************
%
%
% Autor:	Heiner Stroick
%
% Dieses Dokument steht unter einer 
% 			Namensnennung -- Nicht-kommerziell -- Weitergabe unter gleichen Bedingungen 4.0 International
% Lizenz.
%
% Die Bedingungen der Lizenz können unter folgendem Link eingesehen werden: 
% 			http://creativecommons.org/licenses/by-nc-sa/4.0/deed.de
%
%

\documentclass[11pt, a4paper]{article}
\usepackage[dvips, bottom=3cm, top=3cm, right=3cm, left=3cm]{geometry}
\usepackage[utf8]{inputenc}
\usepackage[ngerman]{babel}
\usepackage[babel,german=quotes]{csquotes}
\usepackage[T1]{fontenc}
\usepackage{amssymb}
\usepackage{amsthm}
\usepackage{graphicx}
\usepackage{amsmath}
\usepackage{MnSymbol,wasysym}
\usepackage{mdframed}
\usepackage{eurosym}
\usepackage{float}
\usepackage{tikz}
\usepackage{setspace}
\usepackage{MnSymbol}
\usepackage{wasysym}
\usepackage{calligra}
\usepackage{polynom}			% Für Polynomdivision und Horner-Schema
\usepackage[colorlinks=true,pdfpagelabels,pdfstartview = FitH,bookmarksopen = true,bookmarksnumbered = true,linkcolor = red,plainpages = false,hypertexnames = false,citecolor = black,urlcolor=red]{hyperref}

\hypersetup{
pdftitle = Hinweise zur Korrektur der HM1-Abgaben,
pdfsubject = Mathematik TU Dortmund Höhere Mathematik,
pdfauthor = Heiner Stroick,
}
\usepackage{footnotebackref}			

\setstretch{1.2} 
\setlength\parindent{0pt}

\newcommand{\Lsg}{\mathbb{L}}
\newcommand{\R}{\mathbb{R}}
\newcommand{\C}{\mathbb{C}}
\newcommand{\D}{\mathbb{D}}
\newcommand{\N}{\mathbb{N}}
\newcommand{\Z}{\mathbb{Z}}
\newcommand{\eps}{\varepsilon}
\newcommand{\Var}{\operatorname{Var}} 
\newcommand{\E}{\mathbb{E}} 
\newcommand{\Omicron}{\mathcal{O}}
\newcommand{\RM}[1]{\MakeUppercase{\romannumeral #1{}}}

\makeatletter
\newcommand{\Spvek}[2][r]{%
  \gdef\@VORNE{1}
  \left(\hskip-\arraycolsep%
    \begin{array}{#1}\vekSp@lten{#2}\end{array}%
  \hskip-\arraycolsep\right)}

\def\vekSp@lten#1{\xvekSp@lten#1;vekL@stLine;}
\def\vekL@stLine{vekL@stLine}
\def\xvekSp@lten#1;{\def\temp{#1}%
  \ifx\temp\vekL@stLine
  \else
    \ifnum\@VORNE=1\gdef\@VORNE{0}
    \else\@arraycr\fi%
    #1%
    \expandafter\xvekSp@lten
  \fi}
\makeatother

\title{Hinweise zur Korrektur der HM\RM{1}-Abgaben WS2016 / 2017}
\author{Heiner Stroick, \href{mailto:heiner.stroick@tu-dortmund.de}{heiner.stroick@tu-dortmund.de}}
\date{Stand: \today}




\begin{document}
\maketitle

\begin{mdframed}[linecolor=red]
\begin{center}
{\footnotesize \textbf{Hinweis:} 

Dies ist kein offizielles Dokument und nicht mit der HM-Orga oder den Dozenten der Vorlesung abgesprochen. Es wird auch nicht von offizieller Seite gegengelesen. 

Die neuste Version wird unter diesem Link zum Download angeboten:\\
~\\
\url{https://github.com/hnrstrck/Anmerkungen-Korrekturen-HM1/}\\
~\\
\textcolor[rgb]{1,0,0}{\textbf{Kein Anspruch auf Vollständigkeit und / oder Korrektheit.}}\\
~\\
\textcolor[rgb]{0,0,1}{Fehler gefunden? Ergänzung? Bitte per \href{mailto:heiner.stroick@tu-dortmund.de}{Mail} melden!}}
\end{center}
\end{mdframed}


\section*{Blatt 2}
\begin{itemize}

\item Ihr sollt beim Auswerten der Summen die Summenformeln benutzen. In der Klausur habt ihr keinen Taschenrechner und auch keine Zeit, jeden Summanden aufzuschreiben und diese dann buchstäblich von Hand zu addieren. 

\item Die wichtigsten Summenformeln solltet ihr auswendig kennen (oder zumindest zur eigenen Sicherheit auf dem Zettel stehen haben). Wenn die Summenformeln zwar auf dem Zettel sind, ihr aber nicht wisst, dass es sie gibt, habt ihr leider auch nichts davon.

\item Bei der geometrischen Summenformel darauf achten, dass die Summation bei $k = 0$ beginnt.

\item Fehler bei der Klammersetzung. Achtet darauf, das ist wichtig. Es gilt:
\begin{equation*}
\sum_{k=0}^n (k+1)^2 \quad \neq \quad \sum_{k=0}^n k^2 + 2k + 1
\end{equation*}

\item  Potenzgesetze sollten euch klar sein.

\item konzentrierter Arbeiten: Viele Abschreibfehler, offene Klammern wurden nicht geschlossen, Indizes hießen mal $k$ und mal $l$, manchmal hatten Summen gar kein Argument.

\item Äquivalenzzeichen machen, wenn ihr Gleichungen umformt. 

\end{itemize}



\newpage
\section*{Blatt 3}
\begin{itemize}
\item Bei Nachweisen wie der Gleichheit in der 1) a) betrachtet man entweder zunächst nur die linke Seite und danach nur die rechte Seite (und stellt dann fest, dass sich beide Ausdrücke in denselben umformen lasen) oder leitet den einen aus dem anderen her.

\item Beim bin. Lehrsatz auf den richtigen Exponenten bei $(x + y)^n$ achten.

\item Die Ausdrücke $(1 + \pi)^9$ und $(2 + a)^7$ waren laut ML auszuschreiben. Achtet auf die Vorfaktoren, die sich aus dem Binomialkoeffizienten ergeben. Ich habe das nicht als Fehler gewertet, wurden die Ausdrücke nicht ausgeschrieben. Ein \enquote{Endergebnis} (eine Zahl) ist nicht interessant, ihr habt in der Klausur auch keinen Taschenrechner.

\item Bei der 2) a) haben einige von euch die Fakultäten sehr umständlich umgeformt. Die Bedeutung von $n!$ ist wichtig und muss euch bekannt sein. Ebenfalls müsst ihr Fakultäten kürzen können: Oft ist es einfacher, Faktoren aus $(n+3)!$ rauszuziehen, um es mit $n!$ zu kürzen, als $n!$ zu erweitern, um $(n+3)!$ komplett zu kürzen (auch weniger Fehleranfällig, da evtl. nicht mit einem Kehrwert multipliziert werden muss). Ihr solltet beide Wege beherrschen: Fakultäten ">vergrößern"< (wie kann ich $(n-4)!$ in $n!$ umformen?) und ">verkleinern"< (wie kann ich $(n+4)!$ in $n!$ umformen?).

\item Wichtig: Klammersetzung bei Fakultäten:
\begin{align*}
(4n)! \quad &\neq \quad 4n!\\
(2n + 1)!\quad &\neq \quad 2n + 1!
\end{align*}
Macht euch das an Beispielen klar.

\item Thema Induktion: Die Induktion besteht aus drei Grundbausteinen: Ind.-Anfang (IA), Ind.-Voraussetzung (IV) und Ind.-Schritt / Ind.-Schluss (IS). Die Ind.-Behauptung (IB) ist nur eine Hilfestellung und darf nicht mit dem (IS) verwechselt oder vermischt werden.

\item Zur 4) a): Legt Wert auf den (IA): Wenn ihr solch eine Summe beweisen sollt, setzt $n = 1$ und schreibt die Summe zunächst noch einmal auf. Achtung: Das Argument der Summe hängt nach wie vor von k ab:
\begin{equation*}
\sum_{k=1}^1 k^2
\end{equation*}
Erst dann rechnet ihr die Summe aus. Dann betrachtet ihr in einer \emph{anderen} Rechnung die rechte Seite des zu beweisenden Ausdrucks (ihr könnt am Anfang nicht einfach ein \enquote{$=$} dazwischen schreiben -- ihr wisst ja nicht, ob es wirklich gleich ist. Eine Möglichkeit wäre, ein Fragezeichen über das Gleichheitszeichen zu schreiben.). Wenn bei beidem dasselbe raus kommt, ist der (IA) gezeigt (">Linke Seite = Rechte Seite"<). Ein \enquote{$1 = 1$} ist zu kurz.

\item Auf Klammersetzung bei der (IB) achten: $n$ durch $(n+1)$ (Klammern!) ersetzen. 

\item Es muss beim (IS) $n \rightsquigarrow n + 1$ heißen und nicht $n \longrightarrow n + 1$,$n \longmapsto n + 1$ , $n = n + 1$ oder $n \implies n + 1$.

\item Ihr solltet auch Terme wie $(2n + 1)$ ausklammern können (oder eine Idee haben wie das geht: Polynomdivision).

\item Eine (IV) wie \enquote{Die Aussage gelte.} ist zu ungenau (und falsch). Die Variante \enquote{Die Aussage gelte für $n \in \N$} ist falsch (dann setzt ihr voraus, dass die Aussage für alle $n$ gilt -- dann wäre der ganze Beweis nicht mehr nötig.) 

\item Bei der 4) b) muss im (IA) auch zunächst ein $k$ im Argument stehen:
\begin{equation*}
\prod_{k=2}^2 \frac{k^2}{k^2 - 1}
\end{equation*}
Der Ausdruck
\begin{equation*}
\prod_{k=2}^2 \frac{2^2}{2^2 - 1}
\end{equation*}
ist falsch (siehe Erläuterungen zur Summe oben).

\item Terme wie $(n+1)^2$ würde ich so spät wie möglich ausschreiben. Guckt erstmal, ob man kürzen kann.

\item Bei der 4) c) solltet ihr die ">Teilbarkeit"< noch einmal mathematisch aufschreiben. Die (IV) lautet dabei: \enquote{Der Ausdruck $2^{2n+1}+1$ sei \textbf{für ein} $n \in \N$ durch 3 teilbar, d.h. es gibt zu diesem $n$ eine Zahl $m \in \N$ mit $2^{2n+1}+1 = 3m$.} Macht euch klar, dass das $m$ von $n$ abhängt. 

\item Die (IB) als Hilfestellung für den Beweis sieht so aus: \enquote{Zu zeigen ist, dass der Ausdruck $2^{2(n+1)+1}+1$ für $n \in \N$ durch 3 teilbar ist, d.h. es gibt zu diesem $n$ eine Zahl $\overline{m} \in \N$ mit $2^{2(n+1)+1}+1 = 3\overline{m}$.}. Macht euch klar, dass das $\overline{m}$ von $n$ abhängt. 

\item Gebt beim (IA) und am Ende des (IS) $m$ und $\overline{m}$ an.

\item Auf Klammersetzung beim Einsetzen der (IV) achten. Wenn ihr in einer Induktion die (IV) nicht gebraucht, habt ihr etwas falsch gemacht.

\item Weiterhin gilt: Auf Äquivalenzzeichen achten, diese bedeuten nicht dasselbe wie Gleichheitszeichen. Macht euch den Unterschied klar und wann man was verwendet.

\item Schreibt zu Beginn einer Rechnung die Aufgabe (nicht die Aufgabenstellung \smiley{}) ab. Eine Rechnung mit \enquote{$= \dots$} zu beginnen und dann schon den ersten Umformungsschritt durchgeführt zu haben, ist nicht so schön.

\item Es heißt $n \in \N$, nicht $n ~\varepsilon~ \N$ und auch nicht $n$ \euro ~$\N$.
\end{itemize}











\newpage
\section*{Blatt 4}
\begin{itemize}
\item Zur Erklärung: \enquote{Not.} = Notation (etwas ist falsch und / oder unsauber aufgeschrieben -- meist bei der Angabe der Lösungsmenge).

\item Zur Erklärung: $\Lsg =~?$, $\D =~?$ meint, dass die Angabe der Lösungsmenge / Definitionsbereich fehlt.

\item Zur Erklärung: \enquote{FU} meint \emph{Fallunterscheidung} (meist falsch gemacht, nicht konsistent mit den anderen oder kritische Stellen vergessen).

\item Bei Fallunterscheidungen zu Ungleichungen ist der Ausdruckt, den ihr untersucht, im Fall $< 0$ nicht durch ein $-($\texttt{AUSDRUCK}$)$ zu ersetzten, d.h. für den Fall $x-1 < 0$ ist der Ausdruck $\frac{1}{x-1}$ in
\begin{align*}
\frac{1}{x-1} < 4x
\end{align*}
nicht durch $\frac{1}{-(x-1)}$ zu ersetzten.


\item Es muss $\Lsg = $~\texttt{MENGE} heißen, z.B. $\Lsg = [0, 6)$. Intervalle sind auch Mengen. Viele vergessen das \enquote{=} oder schreiben nur $\Lsg = \left\{\frac{3}{2}\right\}$ (was prinzipiell richtig wäre, aber dann meist die Fallvoraussetzung vergessen wurde -- z.B. $x < 4$. $\Lsg =\left(\frac{3}{2}, 4\right)$ wäre dann richtig).

\item Die Lösungsmenge 
\begin{align*}
\Lsg =\left\{\frac{3}{2} < x < 4\right\}
\end{align*}
ist etwas ungenau, weil da nicht zu erkennen ist, aus welchem Zahlenbereich ($\Z, \N, \R$) das $x$ ist. Besser ist: 
\begin{align*}
\Lsg =\left\{x \in \R \mid \frac{3}{2} < x < 4\right\}
\end{align*}
Oder kürzer (geht nur im Reellen!):
\begin{align*}
\Lsg =\left(\frac{3}{2}, 4\right)
\end{align*}

\item Die Zeichen $\cap$ und $\cup$ sind Zeichen, die zwischen \emph{Mengen} stehen (Schnitt, Vereinigung). 

\textbf{Nicht verwechseln:} Die Zeichen $\land$ und $\lor$ (UND, ODER) stehen zwischen \emph{Aussagen}. Gleichungen oder Ungleichungen sind \emph{Aussagen}. Lösungsmengen sind \emph{Mengen} und keine Aussagen. 

Tipp zum Merken: $\lor$ vommt von lat. \enquote{\textbf{v}el} (\enquote{oder}), $\land$ meint dann eben \enquote{und}.

\item Achtet bei Fallunterscheidungen darauf, wirklich den kompletten Definitionsbereich abzudecken. Wenn ihr mit den Zeichen $<$ und $>$ unsauber umgeht, vergesst ihr Werte (eben die \enquote{Sprungstellen}). Oder ihr nehmt Werte dazu, wo eigentlich Definitionslücken sind.

\item Verwechslungsgefahr: Ihr dürft im Betrag etwas negatives oder 0 haben. Das heißt, eine vorherige Untersuchung für den Definitionsbereich, wo z.B. für $|5x - 6| > 0$ das Argument des Betrages (eben $5x - 6$) Null wird, ist nicht nötig.

Also: Der Definitionsbereich ist hier $\R$, insbesondere $x = \frac{6}{5}$. (Der Wertebereich bzw. die Lösungsmenge sieht natürlich anders aus und ist nicht notwendigerweise immer $\R$; hier auch nicht. Eine Untersuchung von $5x - 6 = 0$ kann darüber hinaus sinnvoll für die Fallunterscheidung sein.) 

\item Wenn ihr aus einem Intervall einzelne Werte ausschließen wollt, sieht die korrekte Notation dafür so aus:
\begin{align*}
\Lsg &= (-3,7) \setminus \{2\}
\end{align*}
Auch okay ist:
\begin{align*}
 \Lsg  = (-3,2) \cup (2,7)
\end{align*}
Etwas wie 
\begin{align*}
\Lsg &= (-3,7) \setminus 2
\end{align*}
ist mathematisch falsch notiert. Der Operator \enquote{$\setminus$} (SETMINUS) steht \emph{stets zwischen Mengen}, und $2$ ist keine Menge, $\{2\}$ aber schon. 

\item Aussagen wie
\begin{align*}
 2 < 2
\end{align*}
sind falsche Aussagen. Es gilt nämlich nur $2 \leq 2$.

Dies kommt bei 1) e) zum Tragen:
\begin{align*}
 0 > -x^2
\end{align*}
ist äquivalent zu $x \neq 0$, da $0 < 0$ falsch ist und sonst das Quadrat einer reellen Zahl immer positiv ist.

\item Es ist $\emptyset \neq \{\emptyset\}$, in der Regel meint ihr ersteres (z.B. bei der Angabe der Lösungsmenge).

\item In der Klausur habt ihr \textbf{keinen Taschenrechner}. Rechnet mit Brüchen und Wurzeln. Zahlen wie $\frac{1}{9} - \sqrt{\frac{4}{7}-6}$ sind völlig okay und müssen auch nicht weiter zusammengefasst werden (Zweierpotenzen bis $2^{10} = 1024$ sollte man auswendig wissen\dots).

\item Auch wenn die Angabe \emph{einer} Lösungsmenge am Ende ausreichend ist ($\Lsg_{Ges}$), solltet ihr überlegen, nach jedem Fall eine Lösungsmenge anzugeben ($\Lsg_1$, $\Lsg_2$, $\dots$). So macht ihr weniger Fehler beim Vereinigen der Lösungsmengen (s. Hinweis zur Notation oben).

\item Es gilt (Intervallgrenzen beachten!)
\begin{align*}
(-\infty, 1) \cup (3, \infty) = \R \setminus [1, 3]
\end{align*}

\item Bei der o) (i) kann kein Intervall angegeben werden, da Punkte in der Ebene (später nennen wir das Vektoren) keine Ordnung besitzen (ähnlich wie komplexe Zahlen kann nicht gesagt werden, welcher Punkt größer ist oder kleiner ist, was aber eine zwingende Voraussetzung für Intervalle ist). Die Lösungsmenge muss anders formuliert werden:
\begin{align*}
\Lsg = \{ (x,y)^\intercal \in \R^2 \mid \qquad \dots \qquad \}
\end{align*}

\item Bei Intervallen steht die kleinere Zahl links des Kommas, sonst ist die durch das Intervall beschriebene Menge leer (Definition der Intervalle im Skript evtl. nochmal ansehen).

\item Ein Komma in den Eigenschaften bei einer Mengenangabe bedeutet \enquote{und}. Das heißt, die folgende Menge ist leer und ist  \textbf{nicht} eine andere Schreibweise für das Intervall $(-\infty, -1) \cup (3, \infty)$:
\begin{align*}
\Lsg = \{ x \in \R \mid x < -1, x > 3\}
\end{align*}
\end{itemize}

















\newpage
\section*{Blatt 5}
\begin{itemize}
\item Beim Zeichnen komplexer Zahlen wird die imaginäre Achse ohne $i$ skaliert. Bei solchen Skizzen solltet ihr (ungefähr) dieselbe Skalierung auf beiden Achsen verwenden (macht das Winkel-Ablesen einfacher!) -- sonst bin ich da nicht so pingelig.

\item Der Imaginärteil einer komplexen Zahl wird \textbf{ohne} $i$ angegeben. Der Imaginärteil ist reell! Das ist ein grober Fehler. Punktabzug.

\item Bei der 2) a) sind $\overline{z}$ oder $z = 0$ nur eine Lösung. Die richtige Idee ist hier, das $z$ durch $x + iy$ zu ersetzen, alles auszumultiplizieren und dann den Imaginärteil zu betrachten. Dieser muss gleich Null gesetzt werden, damit das Produkt reell ist. Dabei reicht es, entweder nach $x$ \emph{oder} nach $y$ umzuformen. Der Realteil ist bereits reell (ein Auflösen nach $x$ bzw. $y$ ist hier nicht nötig).

Grob falsch ist folgender Rechenweg hinter der Idee \enquote{Ein Produkt ist genau dann gleich 0, wenn beide Faktoren Null sind.} Also wäre $z = 0$ oder $3 + 5i = 0$ (bis hierhin noch okay -- die zweite Gleichung ist nicht erfüllt\footnote{Hier steht so etwas wie $3 = 0$ im Reellen.}). Die zweite Gleichung dann aber nach $i$ aufzulösen ist grob falsch, in der Gleichung kommt gar keine Variable vor. Punktabzug. 

Eine Lösungsmenge muss angegeben werden (hier nach $x$ aufgelöst):
\begin{align*}
\Lsg = \{ z \in \C \mid z = x - i\frac{5}{3}x\ \text{~mit~} x \in \R\}
\end{align*}


\item Zwischen einfache Termumformungen kommen Gleichheitszeichen, wenn ihr Gleichungen umformt, kommen Äquivalenzzeichen dazwischen. Punktabzug (ich habe es oft genug geprädigt\dots)

\item Es ist ganz nützlich,
\begin{align*}
z\cdot\overline{z} = |z|^2 = x^2 + y^2
\end{align*}
auswendig zu wissen. Erspart einem auf jeden Fall ein bisschen Arbeit.

\item Bei der 2) c) (i) reicht es nicht, zu erkennen, dass es da ein \enquote{Muster} gibt. Ich hätte gerne so etwas wie
\begin{align*}
i^n = \begin{cases}
1, \quad &n = 4m		\quad  \text{für}\quad m \in \N\\
i, \quad &n = 4m + 1	\quad  \text{für}\quad m \in \N\\
-1, \quad &n = 4m + 2 	\quad  \text{für}\quad m \in \N\\
-i, \quad &n = 4m + 3	\quad  \text{für}\quad m \in \N\\
\end{cases}
\end{align*}
gelesen.

\item Wenn nach dem Ausrechnen des Betrages einer Komplexen Zahl ein $i$ auftaucht, ist das ein grober Fehler. Der Betrag einer komplexen Zahl ist reell! Punktabzug.

Guckt euch nochmal an, wie man in $\C$ den Betrag berechnet. \textbf{Insbesondere ist keine Fallunterscheidung wie im Reellen nötig}.\footnote{Das ist auch allein schon deshalb nicht sinnvoll, da die komplexen Zahlen keine Ordnung besitzen. Das heißt, eine Fallunterscheidung wo $3 - 4i < 0$ ist, ist sinnlos (und nicht lösbar und falsch \smiley{}).} Den Betrag kann man einfach ausrechnen. 

\item Ob man Gleichungen korrekt gelöst hat, kann man überprüfen, indem man die vermeintliche Lösung (bei $\infty$-vielen Lösungen: eine bzw. mehrere) für $z$ (die Variable) einsetzt. Geht die Gleichung auf, hat man richtig gerechnet.

\item Es ist oft sinnvoll, bei Gleichungen das $z$ so spät wie möglich durch $x + iy$ oder $a + ib$ zu ersetzten.

Zum Beispiel würde ich 
\begin{align*}
(z -2i)^2 = (z + 2i)^2
\end{align*}
zunächst mit $z$ ausrechnen, weil so nur zwei Summanden in der Klammer stehen (einfache bin. Formel). Ersetzt man das $z$ durch $x + iy$ am Anfang, hat man drei Summanden (das Potenzieren mit $2$ ist umfangreicher und fehleranfälliger). Auch wenn man alle Summanden mit $i$ zusammenfasst, entstehen neue bin. Formeln, die ausmultipliziert werden müssen.
\begin{align*}
(x + iy -2i)^2 = (x + iy + 2i)^2
\end{align*}

\item Eine Möglichkeit, Lösungsmengen in $\C$ zu notieren (hier für die Aufgabe j). Beachtet, dass es hier $\infty$-viele Lösungen gibt):
\begin{align*}
\Lsg = \{ z = x  + iy \in \C \mid y = -\frac{3}{5}x + \frac{8}{5}\}
\end{align*}

\item Wenn ihr für die Lösung einer Gleichung \enquote{nur} zwei komplexe Zahlen raus bekommt, könnt ihr diese einfach aufzählen:
\begin{align*}
\Lsg_{\text{k) (ii)}} = \{ -2 + 2i, -3 + 3i\}
\end{align*}

\item Bei der k) (i) und (ii) empfiehlt sich eine Skizze. Dabei kommt man bei der (i) nach ein paar Rechenschritten auf eine Kreisgleichung -- (ii) ist dann die Schnittmenge des Kreisbogens um $-3 + 2i$ mit Radius $1$ und der Geraden, die durch $a - ai$ beschrieben wird (Winkelhalbierende des \RM{2}. und \RM{4}. Quadranten).

\item Wenn ihr Gleichungen löst, \textbf{gebt auch eine Lösungsmenge an!} Einfach nur $a = \dots$ und $b = \dots$ irgendwo auf der Seite stehen zu haben ist nicht so schön und unvollständig.

\item Ich habe öfters gelesen, komplexe Zahlen wie $-1$ oder $1$ hätten keinen Imaginärteil. Das ist falsch. Diese Zahlen haben einen Imaginärteil, dieser ist 0. 

Also: $\operatorname{Im}(-1) = 0$. 

\item Nicht vergessen: $x^2 = 1$ hat auch die Lösung $x = -1$.
\end{itemize}




\newpage
\section*{Testat 2}
\begin{itemize}
\item Da kaum jemand $z^8$ richtig bestimmt hat (vielleicht 5 von 60 Leuten), hier ein Lösungsvorschlag. Seht mir bitte nach, dass ich ihn nicht getext habe\dots \smiley{}
\end{itemize}

\begin{figure}[H]
	\centering
	\fbox{\includegraphics[width=1.0\textwidth]{img/testat_2_zhoch8.jpg}}
	\caption{Bestimmen von $z^8$}
	\label{fig:testat2_zhoch8}
\end{figure}


\newpage
\section*{Blatt 6}
\begin{itemize}
\item Wenn eine Zahl $z = x + iy \in \C$ in Polarkoordinatendarstellung (PKD) gebracht werden soll, reicht es nicht, nur den Betrag und den Winkel zu bestimmen. Es muss abschließend noch einmal $z$ in PKD angegeben werden. (Genauso wie am Ende immer noch einmal die Angabe der Gesamtlösungsmenge erwartet wird\dots)

\item Winkel werden in Bogenmaß gemessen (\texttt{RAD}). Winkel wie 360\textdegree, 180\textdegree, 90\textdegree, 60\textdegree~und 45\textdegree~sollte man in Bogenmaß kennen (oder sich eben schnell herleiten können).

Das Gradmaß (\texttt{DEG}) hat in der Mathematik nichts verloren (die Festlegung vom rechten Winkel auf 90\textdegree~ist recht willkürlich. Daneben (und auch deswegen) gibt es noch andere Gradmaße, mit denen man einfacher rechnen kann, z.\,B. \texttt{GON} mit mit einem rechten Winkel von 100~$\operatorname{gon}$)

\item Beim Umformen einer Zahl in PKD solltet ihr getrennt voneinander Winkel und Argument bestimmen (und dann noch einmal $z$ in PKD angeben, vgl.\,Abbildung~\ref{fig:testat2_zhoch8})


\item Bei der 1) b) ist $z_3$ \textbf{nicht} in PKD (man beachte das Minus vor dem Sinus). 

Soll hier mit $z_3$ gerechnet werden, muss das Minus zu einem Plus gemacht werden. 

Dieser Weg klappt immer: $z_3$ in die Form $x + iy$ bringen und dann in PKD umformen. Dann kann man die Rechengesetze für zwei komplexe Zahlen in PKD anwenden (z.\,B. das Rechengesetz für die Multiplikation). 

Geht evtl. schneller: Rechengesetze für $\sin$ und $\cos$ anwenden, um $z_3$ in PKD umzuformen. Achtung: \enquote{Beide} Winkel bei der PKD müssen gleich sein.

Solch eine Aufgabe ist auch auf einem der Trainingsblätter zu den komplexen Zahlen drauf. 

\item Wenn ihr im Argument von $\sin$ und $\cos$ rechnet, \textbf{setzt Klammern!} Sonst kann man nicht nachvollziehen, was ihr meint!
\begin{align*}
\cos \frac{3}{2}\pi - \pi 			&\stackrel{?}{=} 		\cos \left( \frac{3}{2}\pi - \pi  \right) \\
\cos \frac{3}{2}\pi - \pi 			&\stackrel{?}{=}			\cos \left( \frac{3}{2}\pi \right) - \pi 
\end{align*}

\item Für die PKD muss der Winkel aus $[0, 2\pi)$ sein. Das bedeutet insbesondere, dass der Winkel positiv angegeben wird! Sonst handelt es sich nicht um eine Zahl in PKD.

Ferner muss sowohl beim $\sin$ als auch beim $\cos$ \textbf{der gleiche Winkel} im Argument stehen.

\item Die Lösungen beim Wurzelziehen werden $z_0, z_1, \dots, z_{n-1}$ durchnummeriert. Dies sind die Lösungen der Gleichung
\begin{align*}
z^n = w
\end{align*}
mit $z, w \in \C$ und $w$ konkret. 

Eine Angabe von $z_n = z_0$ ist nicht erforderlich -- ich hatte das im Tutorium nur angeschrieben, damit ihr den zyklischen Aufbau versteht. 

\textbf{Tipp:} Zunächst Winkel getrennt von der Aufstellung der Lösungen berechnen. So macht ihr weniger Fehler.

Und gebt $w$ bitte einmal in PKD an, bevor ihr die Lösungen $z_0, z_1, \dots, z_{n-1}$ aufstellt. 


\item Standardwerte von $\sin$ und $\cos$ \textbf{muss} man auswendig wissen. Dazu zählen die Winkel $0, \frac{\pi}{2}$ und $\pi$. Es wird erwartet, dass diese Ausdrücke ausgewertet werden!

\item Macht ein ordentliches $\pi$! Bei einigen sieht das aus wie $n$, $\cap$ oder $\sqcap$.

\item Wenn ihr beim Sinus Klammern setzt, müsst ihr zwei Klammern schließen (die vom Sinus und die vom ausgeklammerten Betrag). Klingt bescheuert, vergessen aber 70\% von euch.

\item Rechnet \textbf{ohne} Taschenrechner. Rechnet mit Brüchen und Wurzeln usw. Tut euch den Gefallen!

\item Die Umkehrfunktion vom $\sin$ wird als Arcus-Sinus ($\arcsin$) bezeichnet. Sie liefert zu einer Zahl den Winkel (auf dem Taschenrechner ist das die Taste \texttt{sin\textasciicircum-1} -- auf das richtige Gradmaß im Taschenrechner achten! \dots aber den dürft ihr ja eh nicht benutzen \smiley{})

In der Mathematik wird die Schreibweise $\sin^{-1}(x)$ für den Arcus-Sinus nicht verwendet, weil man es nicht von $\frac{1}{\sin(x)}$ unterscheiden kann. Schreibt $\arcsin(x)$ für die Umkehrfunktion.

\item Wenn nicht weiter angegeben ist, wie die Lösung einer Rechnung angegeben werden soll, ist ein Ergebnis in PKD natürlich völlig in Ordnung.

\item Die \enquote{gemischte Bruchschreibweise} wird in der Mathematik nicht verwendet, da man nicht unterscheiden kann, ob $5\frac{1}{3}$ nun $5 \cdot \frac{1}{3}$ oder $5 + \frac{1}{3}$ bedeutet. 

\item Nach wie vor gilt: Gebt euch Mühe beim Setzen von Äquivalenz- und Gleichheitszeichen. Und setzt das richtige Zeichen an die richtige Stelle. Arbeitet mathematisch sauber! Sonst: Punktabzug!

Dazu: Wenn ihr das Zeichen $\stackrel{\wedge}{=}$ in der Mathematik verwendet, könnt ihr euch zu 90\% sicher sein, dass ihr es falsch oder unsauber notiert habt.

\item Es gilt: 
\begin{align*}
-64 \in \C.
\end{align*}
Überlegt euch, wie die PKD aussehen muss. Wo liegt die Zahl in der komplexen Ebene? 

\item Beim Bestimmen der PKD ist die Zahl, die ausgeklammert wird (also $|z|$), immer positiv. Dies muss insbesondere beim Bestimmen der PKD von $-64$ beachtet werden. 

Wenn ihr hier $64$ ausklammert, kommt ihr auf den richtigen Winkel von $\alpha = \pi$. 

Klammert ihr $-64$ aus, kommt ihr auf einen Winkel von $\alpha = 0$. Ein Winkel von $0$ hat aber keinen Sinn (überlegt euch, wo $-64$ in der komplexen Ebene liegt). 

\item Erinnerung: Die Vorschrift zum Bestimmen der Winkel für die Lösungen $z_1, z_2, \dots, z_{n-1}$ lautet (es sei $\varphi = \arg(z_k) = \frac{\arg(w)}{n}$ für $k = 0,1,2,\dots,n-1$):
\begin{align*}
\arg(z_k) = \varphi + k \cdot \frac{2\pi}{n} \qquad\text{für}\qquad k = 0,1,2,\dots,n-1
\end{align*}

\item Soll beim Wurzelziehen am Ende eine Lösungsmenge angegeben werden, reicht das aus:
\begin{align*}
\Lsg = \{z_1, z_2, \dots, z_{n-1}\}
\end{align*}

\item $2^5 = 32$, $2^6 = 64$. Zweierpotenzen bis $2^{10}$ sollte man auswendig wissen. 

\vfill

\item Maximal könnt ihr derzeit $20 + 19 + 4\cdot 5 = 59$ Punkte erreicht haben (Testate 1, 2) und (ÜB 2--6). Von insgesamt $20 \cdot 5 + 4 \cdot 13 = 152$ Punkten sind das 38\% und schon fast die Studienleistung, die es bei 60 Punkten gibt. Das heißt: Wer derzeit ca. 15 Punkte (25\% der Studienleistung) hat, \textbf{sollte Gas geben}! Es werden nur noch ca. 100 Punkte ausgeschüttet, von denen ihr dann 45 braucht! 

\end{itemize}









\newpage
\section*{Blatt 7}
\begin{itemize}
\item Kurzes Update: Es gilt lt. Skript $0 \in \R^{+}$ (Notation im Skript: $\R_{+}$).

\item Bei Polynomdivisionen \textbf{müssen} Klammern um den Dividenden gesetzt werden.\footnote{Dividend / Divisor = Quotient} Sonst wird nur das letzte (meist absolute) Glied durch den Divisor geteilt.

\item Kommt eine Nullstelle bei der $pq$-Formel oder sonstwo häufiger vor, taucht sie natürlich in der Linearfaktorzerlegung (LFZ) in entsprechender Potenz vor.

\item Wenn ihr die LFZ angeben sollt, gehört ein $f(x)$ vor die LFZ. Sonst steht der Rest einfach nur im leeren Raum. 

Also z.\,B.:
\begin{align*}
f(x) = -3(x-4)(x+5)(x-1)^7(x-1)^2
\end{align*}

\item Wenn ihr auf dem \enquote{Weg} zur LFZ z.\,B. bei einer Berechnung von der $pq$-Formel durch $2$ teilt (Aufgabe 1) a) iii)), dann dürft ihr die 2 nicht in der LFZ vergessen. Ihr hättet diese ja auch ganz (!) zu Beginn ausklammern können. Macht euch das klar.

\item \textbf{Beispiele sind keine Beweise!} Wenn in Aufgaben etwas wie \enquote{Zeigen Sie, dass \dots} steht, werden Beweise verlangt.\footnote{Die zu zeigende Behauptung stimmt -- im Gegensatz zu Formulierungen wie \enquote{Gilt \dots~für \dots} oder \enquote{Überprüfen Sie, ob \dots~gilt}, wo der Wahrheitsgehalt der Aussage gezeigt / widerlegt werden muss.} Für die Beweise lohnt sich ein Blick ins Skript.

\item Es gilt:
\begin{align*}
\frac{y}{2} = (x+1)^4\\
\Leftrightarrow \sqrt[4]{\frac{y}{2}} = |x + 1|
\end{align*}

Das ist etwas anderes als
\begin{align*}
\sqrt[4]{\frac{y}{2}} = x + 1.
\end{align*}

Siehe dazu auch die Musterlösung für weitere / die restlichen Umformungsschritte. 

\item Die Monotonie sollt ihr nicht über die 1. Ableitung zeigen. Das ist Stoff, der (noch nicht) behandelt wurde. Dies ist Stoff aus der  HM\RM{2}. Arbeitet deshalb mit der Definition im Skript. 

Dazu: $|x|$ ist in 0 nicht differenzierbar (diff'bar). $|x+3$ ist in $x = -3$ nicht diff'bar. Das heißt, man kann keine Ableitung in $x = 0$ bzw. $x = -3$ berechnen (u.\,a deswegen kann man die Monotonie nicht auf ganz $\R$ mit dem Ableitungskriterium zeigen). Das solltet ihr schon mal gehört haben; kommt ausführlich nächstes Semester.

\item Es gilt: $f(x) = |x|$ ist auf $\R$ weder (streng) monoton fallend noch (streng) monoton fallend. Ihr müsst die Funktion \textbf{als ganzes} und nicht nur ihre Äste sehen. Und den Definitionsbereich $\R$ beachten.

Dann tut eine Skizze ihr übriges:

\begin{figure}
\centering
\begin{tikzpicture}[domain=-4:4] 
    \draw[very thin,color=gray] (-3.9,-0.1) grid (3.9,3.9);
    \draw[<->] (-4.2,0) -- (4.2,0) node[right] {$x$}; 
    \draw[->] (0,-0.1) -- (0,4.2) node[above] {$f(x)$};
    \draw[color=red]    plot (\x,{abs(\x)})             node[right] {$f$}; 
\end{tikzpicture}
\end{figure}

\item Wenn man zu Beginn ein $x$ bei einem Polynom ausklammern kann, solltet ihr das tun. Dadurch werden Polynomdivision bzw.\,Horner-Schema kleiner. Macht euch das ggfs.\,klar.

\item Redet ihr von Funktionen, sprecht ihr diese natürlich beim Namen an. Auch wenn man überall immer $f(x)$ ließt, heißt die Funktion $f$. $f(x)$ bezieht sich auf die $y$-Werte (meist für ein oder mehrere \textbf{konkrete} $x$-Werte ($x = 7$, $x = -\pi$, \dots).

Es muss also korrekterweise heißen:
\begin{itemize}
\item $f$ ist streng monoton fallend (nicht $f(x)$)
\item $f$ ist stetig (nicht $f(x)$)
\item Der Graph von $f$ sieht aus wie \dots (nicht der Graph von $f(x)$)
\item $f$ ist differenzierbar (nicht $f(x)$)
\item $f$ ist bijektiv (nicht $f(x)$)
\item Es werden die Hoch- und Tiefpunkte von $f$ bestimmt (nicht von $f(x)$
\item $p$ hat den Grad 4 (nicht $p(x)$)\footnote{Ja, das habe ich letztes Tutorium beim Anschreiben selber nicht so angeschrieben -- mit Absicht. Ich glaube, es ist am Anfang einfacher, sich vorzustellen, dass $p(x)$ den Grad 4 hat, als dass $p$ den Grad 4 hat. $p(x)$ oder $q(x)$ erinnert mehr an ein Polynom als es vielleicht $p$ oder $q$ tut.}
\item usw. Es gibt vielfach mehr solche Beispiele.
\end{itemize}

\item Beim Zeigen von Gleichungen sucht ihr euch eine Seite aus und folgt dann durch einfache Termumformungen die andere Seite zu erhalten. Wenn $A = B$ zu zeigen ist, sehen die beiden Möglichen Lösungen so aus:
\begin{align*}
A = \dots = \dots = \dots = \dots = \dots = \dots = B
\end{align*}
oder
\begin{align*}
B = \dots = \dots = \dots = \dots = A
\end{align*}

Es kann sein, dass mal die Richtung von links nach rechts oder die von rechts nach links einfach ist als die jeweils andere. Da aber eben ein \enquote{$=$} dazwischen steht, sind beide Rechnungen gleichermaßen richtig.

\item Tipp für die Klausur: Aufgabenstellung lesen. Wer keine LFZ angibt, obwohl das in der Aufgabenstellung steht, kriegt Punktabzug. Auch wenn alle Nullstellen richtig sind und von mir aus auch ordentlich alle an einer Stelle angegeben werden. Gleiches gilt für die Angabe einer Lösungsmenge.

Zweiter Tipp: Lest jede Aufgabenstellung zwei mal. Und lest sie noch einmal, wenn ihr glaubt fertig zu sein, um zu checken, ob ihr nichts vergessen habt.

Dazu: Wenn nach Injektivität und Surjektivität gefragt ist, müsst ihr auch beides untersuchen.

\smiley{}

\item Die Aussagen
\begin{align*}
f \text{~surjektiv} \Rightarrow f \text{~nicht injektiv}
\end{align*}
oder 
\begin{align*}
f \text{~injektiv} \Rightarrow f \text{~nicht surjektiv}
\end{align*}
sind falsch. Injektivität und Surjektivität sind keine gegensätzlichen Begriffe und implizieren in keiner Weise einander.

Es gilt aber
\begin{align*}
\left(f \text{~nicht surjektiv} \lor f \text{~nicht injektiv}\right)   \Rightarrow f \text{~nicht bijektiv}.
\end{align*}

\item Es gilt im allgemeinen nur die Implikation 
\begin{align*}
f \text{~streng monoton} \Rightarrow f \text{~injektiv}.
\end{align*}
Die andere Richtung (Rückrichtung)
\begin{align*}
f \text{~streng monoton} \Rightarrow f \text{~injektiv}.
\end{align*}
ist falsch! Einige wollten damit die 3) a) ii) lösen. 

Ein Gegenbeispiel ist die Funktion $h \colon [0,2] \rightarrow [0,2]$ mit 
\begin{align*}
h(x) = \begin{cases}
-x+1 \quad  \text{für}\quad x \in [0,1)\\
x \quad \text{für} \quad  x \in [1,2]\\
\end{cases}
\end{align*}
Zeichnet euch die Funktion auf. $h$ ist injektiv (sogar bijektiv!), aber nicht streng monoton steigend oder steigend (auf $[0, 2]$ -- vgl.\,Überlegungen zur Betragsfunktion weiter oben). Das liegt daran, dass $h$ nicht stetig ist.

Ich habe das nicht so streng korrigiert, weil ihr die Stetigkeit der Funktion in der Aufgabe implizit vorausgesetzt habt (wenn anscheinend auch unbewusst\dots). Für stetige Funktionen gilt in obigen Implikationen nämlich $\Leftrightarrow$.

\end{itemize}




\newpage
\section*{Blatt 8}
\begin{itemize}
\item Wie bereits öfters erwähnt, wird die Angabe einer Lösungsmenge beim Lösen von Gleichungen verlangt. Und $Ax = b$ ist eine Gleichung.

\item Ihr dürft bei der 1) auch die Schritte a)--d) für jedes LGS zusammen machen. Dann bleibt man in der Aufgabe.

\item \textbf{Schreibt beim Umformen eines LGS daneben, was ihr rechnet bzw. welche Zeile ihr \emph{wie} manipuliert.} Die Notation sollte wie im Tutorium besprochen erfolgen. 

\item Ihr solltet für Parameter $s$ und $t$ schreiben und nicht die Variable nutzen. Das heißt, ihr gewinnt aus einer Zeile die Information, dass  $x_3$ beliebig ist, und solltet dann $x_3 =: t$ notieren (und mit $t$ für $x_3$ weiter rechnen).

\item Die Lösungsmenge wird bei $\infty$-vielen Lösungen wie im Tutorium wie folgt angegeben. Ihr geht vom groben (erst alle Vektoren $x \in \R^3$) ins Feine ($x$ soll solche Gestalt haben: \dots). Eine andere Richtung ist nicht sinnvoll.
\begin{align*}
\Lsg = \left\{ x \in \R^3 \mid x = \Spvek[c]{1;1;0} + t\Spvek[c]{-1;1;1}, t \in \R\right\}. 
\end{align*}

Ist nur \emph{ein} Vektor in der Lösungsmenge, kann dieser einfach angegeben werden, z.\,B.:
\begin{align*}
\Lsg = \left\{\Spvek[c]{\frac{1}{7};1;-2}\right\}. 
\end{align*}

Nicht die Klammern um den Vektor vergessen! Und es heißt $\Lsg = \{\dots\}$, nicht $\Lsg : \{\dots\}$
\item Wenn etwas wie \enquote{Ist $v \in \Lsg$?} gefragt ist, wird auch ein Antwortsatz erwartet. Ich habe dann \enquote{Aussage?} daneben geschrieben.

\item Der Vektor in 2) b) ii) heißt $v$. $v^\intercal$ ist nur eine verkürzte Schreibweise eines Spaltenvektors als Zeile (damit das im Text nicht so viel Platz weg nimmt). 

Also: Es wird überprüft, ob $v$ in $\Lsg$ ist, nicht ob $v^\intercal \in \Lsg$ gilt.

\item Fallen bei der Vorwärtselimination Zeilen raus (weil sie identisch sind), schreibt für diese Zeile $0 = 0$ und lasst diese nicht einfach weg.\footnote{Das Weglassen wäre mathematisch nicht falsch, es ist aber leichter für die Korrektur, wenn die Zeilen stehen bleiben.} 

Beispiel:
\begin{align*}
\left(
\begin{array}{ccc|c}
1 & 3 & 4 & 1\\
1 & 5 & 7 & 1\\
1 & 3 & 4 & 1\\
\end{array}
\right)
&
\Leftrightarrow
\left(
\begin{array}{ccc|c}
1 & 3 & 4 & 1\\
  & 2 & 3 & 0\\
  &   & 0 & 0\\
\end{array}
\right)
\end{align*}
Eher nicht (s.\, Fußnote): 
\begin{align*}
\left(
\begin{array}{ccc|c}
1 & 3 & 4 & 1\\
1 & 5 & 7 & 1\\
1 & 3 & 4 & 1\\
\end{array}
\right)
&
\Leftrightarrow
\left(
\begin{array}{ccc|c}
1 & 3 & 4 & 1\\
  & 2 & 3 & 0\\
\end{array}
\right)
\end{align*}

\item Zwischen die Umformungsschritte bei der Vorwärtselimination gehört ein \enquote{$\Leftrightarrow$} oder \enquote{$\rightsquigarrow$}. Ganz bestimmt aber nicht \enquote{$=$}, weil die Schreibweise $\left(A|b\right)$ ja schon eine verkürzende Schreibweise für Gleichungen ist. 

\item Es gibt mehrere Wege, die nach Rom führen (bzw\,zur ZSF) \smiley{}

Wenn die ZSF angegeben werden soll (Aufgabe 1), solltet ihr das auch tun. Auch im Hinblick auf die (Probe-)klausur.

Die Matrix
\begin{align*}
\left(
\begin{array}{ccc|c}
1 & 3 & 4 & 1\\
1 & 0 & 0 & 8\\
0 & 3 & 4 & 4\\
\end{array}
\right)
\end{align*}
ist nicht in ZSF (auch wenn man die Lösungsmenge sehr leicht bestimmen kann: \RM{2}~$\rightarrow$~\RM{3}~$\rightarrow$~\RM{1}).

\item Nur, dass ich es mal erwähnt habe:
\begin{align*}
\Spvek[c]{0;1;-2;-5} \in \R^4 \qquad \Spvek[c]{0;1;-2;-5;9} \in \R^5
\end{align*}

\item $\operatorname{rang}(A) > \operatorname{rang}(A|b)$ ist nicht möglich (klar machen!). Notation ist wie folgt: $\operatorname{rang}(A) = 3$, nicht $\operatorname{rang}(A) : 3$.

\end{itemize}

\vspace{1cm}















\begin{center}
\fontsize{18pt}{18pt}{{\calligra Fröhliche Weihnachten und einen guten Rutsch!}}
\end{center}



%
%
%
%
%\newpage
%\section*{Probeklausur Dezember 2016}
%\subsection*{Aufgabe 1}
%\begin{itemize}
%\item 
%\end{itemize}
%
%\subsection*{Aufgabe 2}
%\begin{itemize}
%\item 
%\end{itemize}




%Lizenz
\newpage
~
\vfill
{\footnotesize
\subsection*{Lizenz}
Dieses Dokument steht unter einer ">Namensnennung -- Nicht-kommerziell -- Weitergabe unter gleichen Bedingungen 4.0 International"<-Lizenz.\\

\noindent
Die Bedingungen der Lizenz können unter folgendem Link eingesehen werden: 

\noindent
\url{http://creativecommons.org/licenses/by-nc-sa/4.0/deed.de}}
\end{document}